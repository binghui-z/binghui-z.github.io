% !TEX TS-program = xelatex
% !TEX encoding = UTF-8 Unicode
% !Mode:: "TeX:UTF-8"
% https://zhuanlan.zhihu.com/p/71432461
\documentclass{resume}
\usepackage{zh_CN-Adobefonts_external} % Simplified Chinese Support using external fonts (./fonts/zh_CN-Adobe/)
\usepackage{linespacing_fix} % disable extra space before next section
\renewcommand{\baselinestretch}{1.2}

\begin{document}
\pagenumbering{gobble} % suppress displaying page number

\name{左炳辉}

\basicInfo{
  \email{binghuizuo@gmail.com} \textperiodcentered\ 
  \phone{(+86) 178-5428-7665} \textperiodcentered\ 
  \faHome{\href{https://binghui-z.github.io/}{Homepage}}%\textperiodcentered\ \
  % \faGoogle{\href{https://scholar.google.com/citations?hl=zh-CN&user=YajLTJMAAAAJ}{Google}}
  }

%%%%%%  教育背景
\section{\faGraduationCap\ 教育背景}
\textbf{东南大学} (2021 -- 至今)  \hspace{3cm}电子信息  \hspace{3cm}导师:王雁刚(副教授)

\textbf{青岛理工大学} (2017 -- 2021)  \hspace{2.22cm}自动化  \hspace{3.41cm}专业排名:1/161(推免)

%%%%%%  参加的项目及论文
\section{\faUsers\ 项目经历}

\subsection{\textbf{Reconstructing Interacting Hands with Interaction Prior from Monocular Images}}
IEEE International Conference on Computer Vision (ICCV2023), 第一作者, (已投稿)
\begin{itemize}
  \item 基于半监督学习框架恢复降质图像的受损外观
  \item 基于ViT,从输入图像中描述裸手的结构
  \item 采用对抗学习策略,学习手的外观细节
\end{itemize}

\subsection{\textbf{Semi-supervised Hand Appearance Recovery via Structure Disentanglement and Dual Adversarial Discrimination}}
International Conference on Image and Graphics (ICIG2023), 第一作者, (已投稿)
\begin{itemize}
  \item 基于半监督学习框架恢复降质图像的受损外观
  \item 基于ViT,从输入图像中描述裸手的结构
  \item 采用对抗学习策略,学习手的外观细节
\end{itemize}

\subsection{\textbf{Semi-supervised Hand Appearance Recovery via Structure Disentanglement and Dual Adversarial Discrimination}}
IEEE Conference on Computer Vision and Pattern Recognition (CVPR2023), 第二作者
\begin{itemize}
  \item 解决了数据集中由于遮挡导致的图像外观被破坏问题,以半监督学习的准则对降质图像进行修复;
  \item 将ViT作为主干网络,先从降质图像中勾勒出手的结构特征,结构特征用法线表示。为了提高精度,我们还利用大量的数据集训练结构先验,即便遮挡严重,也能够估计准确的结构特征。
  \item 采用对抗学习策略,将手的外观细节映包裹到结构特征上。此过程中,我们结合了pix2pix和cycleGAN的优缺点,提出了一种双重对抗策略,使得外观恢复的表现更加鲁棒。
\end{itemize}

\subsection{\textbf{Stability-driven Contact Reconstruction From Monocular Color Images}}
IEEE Conference on Computer Vision and Pattern Recognition(CVPR2022),  第二作者

\begin{itemize}
  \item 使用基于物理引擎的方法进行手物交互重建,并通过物理指标对重建过程进行指导
  \item 使用椭球表示手和物体的重建结果,有利于物理仿真过程
  \item 提出了一个多视角、大规模、含有物理指标的手物交互数据集
\end{itemize}

\subsection{\textbf{多视角同步手势采集系统}}
\begin{itemize}
  \item 单反相机和工业相机均匀分布,可用于图像或视频序列数据集的拍摄
  \item 通过硬件触发的方式实现相机同步拍摄,用于获得高精度的同步数据
  \item 系统可扩展性强,视角密度大,同步性能高
\end{itemize}

\subsection{\textbf{专利:一种基于物理引擎的三维手物交互重建方法和系统}}
\vspace{2mm}

%%%%%%  掌握的技能
\section{\faCogs\ 技能}
\begin{itemize} [parsep=0.5ex]
  \item 编程语言:Python , C++
  \item 框架与常用库:PyTorch, OpenCV, Eigen, Open3d
  \item 其他技能:阅读英文文献与技术文档,Latex,Git
\end{itemize}

%%%%%%  荣誉称号吗&奖学金
\section{\faTrophy\ 获奖情况}
\begin{itemize}
    \item{\datedline{三等学业奖学金}{2022.10}}
    \item{\datedline{山东省优秀毕业生}{2021.7}}
    \item{\datedline{山东省优秀学生}{2020.5}}
    \item \datedline{国家奖学金}{2019.9}
\end{itemize}

\end{document}
