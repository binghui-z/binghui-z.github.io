% !TEX TS-program = xelatex
% !TEX encoding = UTF-8 Unicode
% !Mode:: "TeX:UTF-8"
% https://zhuanlan.zhihu.com/p/71432461
\documentclass{resume}
\usepackage{zh_CN-Adobefonts_external} % Simplified Chinese Support using external fonts (./fonts/zh_CN-Adobe/)
\usepackage{linespacing_fix} % disable extra space before next section
\renewcommand{\baselinestretch}{1.2}

\begin{document}
\pagenumbering{gobble} % suppress displaying page number

\name{左炳辉}

\basicInfo{
  \email{~binghuizuo@gmail.com} \textperiodcentered\ 
  \phone{~(+86) 178-5428-7665} \textperiodcentered\ 
  \faHome{\href{https://binghui-z.github.io/}{~Homepage}}%\textperiodcentered\ \
  % \faGoogle{\href{https://scholar.google.com/citations?hl=zh-CN&user=YajLTJMAAAAJ}{Google}}
  }

%%%%%%  教育背景
\section{\faGraduationCap\ 教育背景}
\textbf{东南大学} (2021 -- 至今)  \hspace{3cm}电子信息  \hspace{3cm}导师:王雁刚(副教授)

\textbf{青岛理工大学} (2017 -- 2021)  \hspace{2.22cm}自动化  \hspace{3.41cm}专业排名:1/161(推免)

%%%%%%  参加的项目及论文
\section{\faUsers\ 项目经历}

\subsection{\textbf{Reconstructing Interacting Hands with Interaction Prior from Monocular Images}}
IEEE International Conference on Computer Vision (ICCV2023), 第一作者, (已投稿)
\begin{itemize}
  \item 针对双手交互遮挡严重的问题,提出了一种新的关节热图表达方式。比起传统的方法,我们的方法能够将交互区域的关节同时映射到热图上;
  \item 现有的方法往往是在interhand2.6M数据集上以images-paired的方式训练,会导致泛化性受限制。我们通过构建与图像无关的交互先验,能够利用更多的相关数据集,泛化性得到提升;
  \item 构建了一个专门面向双手交互任务的数据集Twohand-500K,该数据集更加多样,更加合理。
\end{itemize}
% \vspace{-2mm}

\subsection{\textbf{Semi-supervised Hand Appearance Recovery via Structure Disentanglement and Dual Adversarial Discrimination}}
IEEE Conference on Computer Vision and Pattern Recognition (CVPR2023), 第二作者
\begin{itemize}
  \item 解决了数据集中由于遮挡导致的图像外观被破坏问题,以半监督学习的准则对降质图像进行修复;
  \item 将ViT作为主干网络,先从降质图像中勾勒出手的结构特征,结构特征用法线表示。为了提高精度,我们还利用大量的数据集训练结构先验,即便遮挡严重,也能够估计准确的结构特征。
  \item 采用对抗学习策略,将手的外观细节映包裹到结构特征上。此过程中,我们结合了pix2pix和cycleGAN的优缺点,提出了一种双重对抗策略,使得外观恢复的表现更加鲁棒。
\end{itemize}
% \vspace{-2mm}

\subsection{\textbf{Stability-driven Contact Reconstruction From Monocular Color Images}}
IEEE Conference on Computer Vision and Pattern Recognition (CVPR2022),  第二作者
\begin{itemize}
  \item 提出了一种手物交互重建的新规范。借助物理引擎平台对交互的合理性进行评判;并通过迭代采样的方式对不合理的交互进行校正;
  \item 为了便于碰撞检测,进而判断手物交互过程是否会出现穿模,我们将手和物体用椭球体表示。此外,这种表达方式具备运动学特征,能够更好的模拟交互;
  \item 提出了一个多视角的大规模数据集,该数据集包含准确的姿态标注,同时含有相关的物理指标。
\end{itemize}

% \subsection{\textbf{Implicit Representation for Interacting Hands Reconstruction from Monocular Color Images}}
% International Conference on Image and Graphics (ICIG2023), 第一作者, (已投稿)

% \subsection{\textbf{专利:一种基于物理引擎的三维手物交互重建方法和系统}}
% \vspace{-1mm}

%%%%%%  掌握的技能
\section{\faCogs\ 个人技能}
研究兴趣包括计算机视觉和计算机图形学,具体研究方向为三维手部重建,尤其是双手交互重建。
\begin{itemize} [parsep=0.5ex]
  \item 编程语言:Python, C++
  \item 框架与常用库:PyTorch, OpenCV, Eigen, Open3d
  \item 其他技能:Latex,Git,Photoshop,Premiere
\end{itemize}

%%%%%%  荣誉称号吗&奖学金
\section{\faTrophy\ 获奖情况}
\begin{itemize}
    \item{\datedline{二等学业奖学金}{2022.10}}
    % \item{\datedline{山东省优秀毕业生}{2021.07}}
    % \item{\datedline{山东省优秀学生}{2020.05}}
    \item \datedline{国家奖学金}{2019.09}
\end{itemize}

\end{document}
